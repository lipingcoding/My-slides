\documentclass[10pt]{beamer}

\usepackage[utf8]{inputenc}
\usepackage{xeCJK}
\usepackage{graphicx}
\usepackage {mathtools}
\usepackage{utopia} %font utopia imported
\usetheme{CambridgeUS}
\usecolortheme{dolphin}
\usepackage{mathtools}
\newcommand{\defeq}{\vcentcolon=}

% set colors
\definecolor{myNewColorA}{RGB}{126,12,110}
\definecolor{myNewColorB}{RGB}{165,85,154}
\definecolor{myNewColorC}{RGB}{203,158,197}
\setbeamercolor*{palette primary}{bg=myNewColorC}
\setbeamercolor*{palette secondary}{bg=myNewColorB, fg = white}
\setbeamercolor*{palette tertiary}{bg=myNewColorA, fg = white}
\setbeamercolor*{titlelike}{fg=myNewColorA}
\setbeamercolor*{title}{bg=myNewColorA, fg = white}
\setbeamercolor*{item}{fg=myNewColorA}
\setbeamercolor*{caption name}{fg=myNewColorA}
\usefonttheme{professionalfonts}
\usepackage{natbib}
\usepackage{hyperref}
%------------------------------------------------------------
\titlegraphic{
\includegraphics[height=1.cm]{ia_icon.pdf}
\hspace{0.5cm}
\includegraphics[height=1.cm]{cripac_icon.pdf}
}

\setbeamerfont{title}{size=\large}
\setbeamerfont{subtitle}{size=\small}
\setbeamerfont{author}{size=\small}
\setbeamerfont{date}{size=\small}
\setbeamerfont{institute}{size=\small}
\title[CRIPAC]{Stable Learning}
\subtitle{Foundations and Applations}
\author[Liping Wang]{Liping Wang}

\institute[wangliping2019@ia.ac.cn]{UCAS, CRIPAC}
\date[Institute of Autation 2022.4]
{Madrid 2021.4}

%------------------------------------------------------------
%This block of commands puts the table of contents at the 
%beginning of each section and highlights the current section:
%\AtBeginSection[]
%{
%  \begin{frame}
%    \frametitle{Contents}
%    \tableofcontents[currentsection]
%  \end{frame}
%}
\AtBeginSection[]{
  \begin{frame}
  \vfill
  \centering
  \begin{beamercolorbox}[sep=8pt,center,shadow=true,rounded=true]{title}
    \usebeamerfont{title}\insertsectionhead\par%
  \end{beamercolorbox}
  \vfill
  \end{frame}
}
%------------------------------------------------------------

\begin{document}

%The next statement creates the title page.
\frame{\titlepage}
\begin{frame}
\frametitle{Contents}
\tableofcontents
\end{frame}
%------------------------------------------------------------
%\section{Introduction}
%    \begin{frame}{Introduction}
%    \end{frame}
\section{Background Knowledge}

\begin{frame}{Hilbert Space \& Kernel}
\begin{definition}[Hilbert Space]
	A Hilbert space is a real or complex \textbf{inner product space} that is also a \textbf{complete} metric space with respect to the distance function induced by the inner product.
\end{definition}

Motivation: to generalize methods of linear algebra and calculus from the finite-dimensional Euclidean spaces to infinite-dimensional spaces.

\begin{definition}[Kernel]
Let $\chi$ be a non-empty set. 
A function $k: \chi \times \chi\rightarrow \mathbb{R}$ is called a kernel if there exists an $\mathbb R$-Hilbert space and a map $\phi:\chi \rightarrow \mathcal H$ such that $\forall x_1, x_2\in chi$, 
\begin{equation}
	k(x_1, x_2) \defeq \langle\phi(x_1),\phi(x_2)\rangle_{\mathcal H}
\end{equation}   
\end{definition}
Motivation: to map features to an infinite-dimensional space.

\end{frame}

\begin{frame}{RKHS(Reproducing Kernel Hilbert Spaces)}




\end{frame}




\section{Conclusion}
    \begin{frame}{Conclusion}
    \end{frame}



\section*{Acknowledgement}  
\begin{frame}
\textcolor{myNewColorA}{\Huge{\centerline{Thank you!}}}
\end{frame}

\end{document}


